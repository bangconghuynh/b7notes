\tikzsetexternalprefix{./general/tikz/}

\section{Preliminaries}
	\label{sec:prelim}

	We introduce several mathematical preliminaries in this \namecref{sec:prelim} to prepare for a more rigorous treatment in what will follow.
	
	In this \namecref{sec:prelim}, we write $\symbb{F}$ to denote a field of numbers that can be either $\symbb{C}$ or $\symbb{R}$.

	\subsection{Maps}
	
		Maps enable objects from different sets or collections to be associated with one another.
		In our study of groups and their representations, we will very often have to relate elements of various different groups together.
		A clear understanding of maps allows us to think about group relationships more logically.

		\begin{definition}[Map]
			Let $X$ and $Y$ be sets.
			A \emph{map} $f: X \rightarrow Y$ is a rule that associates to each element $x \in X$ a unique element $f(x) \in Y$.
			We can then write $f: x \mapsto f(x)$.
			$X$ and $Y$ are called the \emph{domain} and the \emph{target} of the map $f$, respectively.
		\end{definition}

		\begin{example}\leavevmode
			\begin{enumerate}[label=(\alph*)]
				\item The map $f: \symbb{R} \rightarrow \symbb{C}$ defined by $f: x \mapsto x+ix$ sends each real number $x$ to a unique complex number $x+ix$.
				\item The map $f: \symbb{R} \times \symbb{R} \rightarrow \symbb{C}$ defined by $f: (x, y) \mapsto x+iy$ sends each ordered pair of real numbers $(x, y)$ to a unique complex number $x+iy$.
			\end{enumerate}
		\end{example}

		\begin{definition}[Injective map]
			A map $f: X \rightarrow Y$ is \emph{injective} if it sends unique elements of $X$ to unique elements of $Y$.
			This means that, for all $x, x' \in X$, $x \neq x' \implies f(x) \neq f(x')$, or equivalently, $f(x) = f(x') \implies x = x'$.
		\end{definition}

		\begin{definition}[Surjective map]
			\label{def:surjective}
			A map $f: X \rightarrow Y$ is \emph{surjective} if, for each $y \in Y$, there is some $x \in X$ such that $f(x) = y$.
		\end{definition}

		\begin{example}\leavevmode
			\label{exp:surjective}
			\begin{enumerate}[label=(\alph*)]
				\item The map $f: \symbb{R} \rightarrow \symbb{C}$ with $x \mapsto x+ix$ is injective because $x+ix = x'+ix' \implies x = x'$.
				However, it is not surjective because there is no $x \in \symbb{R}$ that gets mapped to $(1+2i) \in \symbb{C}$.
				\item \label{exp:xtox2} The map $f: \symbb{R} \rightarrow \symbb{R^+_0}$ with $x \mapsto x^2$ is not injective because $-2 \neq 2$ but $(-2)^2 = 2^2 = 4$.
				However, it is surjective because every non-negative real number $y \in \symbb{R^+_0}$ must be the square of some real number.
			\end{enumerate}
		\end{example}
	
		\begin{definition}[Bijective map]
			A map $f: X \rightarrow Y$ is \emph{bijective} if it is both surjective and injective, \textit{i.e.}, $f$ sends every $x \in X$ to one and only one $y \in Y$, and every $y \in Y$ has exactly one $x \in X$ that gets sent to it.
		\end{definition}
	
		\begin{example}
			The map $f: \symbb{R} \rightarrow \symbb{R}$ with $x \mapsto 5x+2$ is bijective.
		\end{example}
	
		\begin{definition}[Image]
			\label{def:image}
			Consider a map $f: X \rightarrow Y$.
			The set $\{y \in Y : y = f(x)\ \textrm{for all}\ x \in X\}$ is a subset of $Y$ called the \emph{image} of $f$.
			We denote this set $\im f$ (the green area in \cref{fig:maps}).
			
			Combining this \namecref{def:image} with \cref{def:surjective}, we see that $f$ is surjective if and only if $\im f = Y$.
			
			\begin{figure}[h!]
				\centering
				\useexternalfile{maps}
				\caption{Image of the map $f: X \rightarrow Y$.}
				\label{fig:maps}
			\end{figure}
		\end{definition}
		
		\begin{example}
			The image of the map $f: \symbb{R} \rightarrow \symbb{R}$ with $x \mapsto x^2$ is $\im f = \symbb{R}^+_0$.
			Clearly $\symbb{R}^+_0 \subset \symbb{R}$ and $f$ is not surjective.
			But if we restrict the target of $f$ to $\symbb{R}^+_0$, then, as in \cref{exp:surjective}\labelcref{exp:xtox2}, $f$ is surjective.
		\end{example}


	\subsection{Groups}
	
		You should be very familiar with all of the following definitions.
		However, they are included here, together with a few interesting examples, for completeness.
		
		\begin{definition}[Binary operation]
			A \emph{binary operation} on a set $X$ is a map $\ \star: X \times X \rightarrow X$ that combines two elements in $X$ to give another element in $X$.
		\end{definition}

		\begin{definition}[Group]
			A \emph{group} is a triple $(G, \star, e)$ of a set $G$, a binary operation $\star$ on $G$, and an element $e \in G$ such that
			\begin{enumerate}[label=(\roman*)]
				\item (closure) for all $a, b \in G$, $a \star b \in G$;
				\item (associativity) for all $a, b, c \in G$, $(a \star b) \star c = a \star (b \star c)$;
				\item (identity) for all $a \in G$, $a \star e = e \star a = a$;
				\item (inverse) for all $a \in G$, there exists some $b \in G$ such that $a \star b = b \star a = e$.
			\end{enumerate}
		\end{definition}
	
		\begin{remark}
			We will write $G$ to denote the group $(G, \star, e)$ whenever the binary operation $\star$ and the identity element $e$ is clear from context.
		\end{remark}

		\begin{definition}[Abelian group]
			A group $(G, \star, e)$ is called \emph{abelian} if $a \star b = b \star a$ for all $a, b \in G$.
		\end{definition}
	
		\begin{example}
			The point groups $\symcal{C}_{2v}, \symcal{D}_{2h}$, and all cyclic groups (\textit{e.g.}, $\symcal{C}_{3}$) are some notable abelian groups.
		\end{example}
	
		\begin{definition}[Group presentation]
			A \emph{presentation} of a group $(G, \star, e)$, denoted by $\braket{S \mid R}$, consists of a set $S$ of \emph{generators} and a set $R$ of \emph{relations} amongst these generators.
			Every element in $G$ can be written as a product of powers of some of these generators, with respect to the binary operation $\star$.
		\end{definition}
	
		\begin{example}\leavevmode
			\begin{enumerate}[label=(\alph*)]
				\item The cyclic group $\symcal{C}_{n}$ has presentation $\braket{r \mid r^n = e}$.
				In point-group symmetry, we identify $r$ with an $n$-fold rotation about some axis $C_n$.
				\item The dihedral groups $\symcal{D}_{nh}$ has presentation $\braket{r, s, t \mid r^n = s^2 = t^2 = e}$.
				In point-group symmetry, we identify $r$ with an $n$-fold rotation about some axis $C_n$, $s$ a two-fold rotation about an axis $C_2 \perp C_n$, and $t$ a reflection in a plane $\sigma_h \perp C_n$.
			\end{enumerate}
		\end{example}
	
		\begin{definition}[Group finiteness and order]
			A group $(G, \star, e)$ is called \emph{finite} if it has a finite number of elements.
			Otherwise, it is called \emph{infinite}.
			We write $|G|$ (or commonly $h$ in Physics and Chemistry) for the number of elements of $G$ and call this the \emph{order} of the group.
		\end{definition}
	
		\begin{example}\leavevmode
			\begin{enumerate}[label=(\alph*)]
				\item The cyclic group $\symcal{C}_{n} = \left(\braket{r \mid r^n = e}, \star, e\right)$ has order $n$ and is finite if $h$ is finite.
				\item There is, up to isomorphism (see \cref{def:isomorphism}), only one unique group of order 3, namely the cyclic group $\symcal{C}_{3}$.
				\item There are, up to isomorphism, only two groups of order 4, namely the cyclic group $\symcal{C}_{4}$ and Klein four-group.
				Klein four-group is isomorphic to the more familiar point groups $\symcal{C}_{2v}$, $\symcal{C}_{2h}$, and $\symcal{D}_{2}$.
				\item $\symcal{C}_{\infty v}$ and $\symcal{D}_{\infty h}$ are two infinite axial point groups most commonly encountered in Chemistry.
				\item The group of all rotations in three dimensions, denoted $\symup{SO}(3)$, is an infinite group.
				The group of all reflections and rotations in three dimensions, denoted $\symup{O}(3)$, is also an infinite group.
			\end{enumerate}
		\end{example}


	\subsection{Group Homomorphisms}

		Group homomorphisms are structure-preserving maps that relate elements from different groups to each other.
		These play a central role in the study of group and representation theory.
		
		\begin{definition}[Group homomorphism]
			Let $(G, \star_G, e_G)$ and $(H, \star_H, e_H)$ be groups.
			A map $\varphi: G \rightarrow H$ is called a \emph{group homomorphism} if, for all $a, b \in G$,
				\begin{equation*}
					\varphi(a \star_G b) = \varphi(a) \star_H \varphi(b).
				\end{equation*}
		\end{definition}
	
		\begin{remark}
			$\varphi$ is essentially a map sending elements in one group, say, $G$, to elements in another group, say, $H$.
			This map is, however, slightly more special because it preserves the structure of the domain group.
			In particular,
			\begin{enumerate}[label=(\roman*)]
				\item it preserves closure: for all $a, b \in G$,
					\begin{equation*}
						\varphi(a) \star_H \varphi(b) = \varphi(a \star_G b) \in \im \varphi\ \textrm{because}\ a \star_G b \in G,
					\end{equation*}
					\textit{i.e.}, the product $\varphi(a) \star_H \varphi(b)$ stays in the image of the map $\varphi$;
				\item it preserves associativity: for all $a, b, c \in G$,
					\begin{align*}
						[\varphi(a) \star_H \varphi(b)] \star_H \varphi(c) %
						&= [\varphi(a \star_G b)] \star_H \varphi(c)\\
						&= \varphi[(a \star_G b) \star_G c]\\
						&= \varphi[a \star_G (b \star_G c)]\\
						&= \varphi(a) \star_H [\varphi(b \star_G c)]\\
						&= \varphi(a) \star_H [\varphi(b) \star_H \varphi(c)];
					\end{align*}
				\item it maps the identity in $G$ to the identity in $H$: for all $a \in G$,
					\begin{equation*}
						\varphi(a) = \varphi(e_G \star_G a) = \varphi(e_G) \star_H \varphi(a) %
						\implies \varphi(e_G) = e_H;
					\end{equation*}
				\item it maps inverse to inverse: for all $a \in G$,
					\begin{equation*}
						e_H = \varphi(e_G) = \varphi(a \star_G a^{-1}) = \varphi(a) \star_H \varphi(a^{-1}) %
						\implies \varphi(a^{-1}) = \varphi(a)^{-1}.
					\end{equation*}
			\end{enumerate}
		\end{remark}
	
		\begin{example}\leavevmode
			\label{exp:homomorphism}
			\begin{enumerate}[label=(\alph*)]
				\item The map $\varphi: G \rightarrow H$ defined by $\varphi: g \mapsto e_H$ is a group homomorphism.
				It is, however, not injective if $|G| > 1$ because different elements in $G$ get mapped to the same element $e_H$ in $H$.
				\item \label{exp:homomorphism.exp} Let $G = (\symbb{R}, +, 0)$ be the additive group of real numbers and $H = (\symbb{R}^+, \times, 1)$ the multiplicative group of positive real numbers.
				The map $\varphi: G \rightarrow H$ defined by $\varphi: g \mapsto \exp(g)$ is a group homomorphism.
			\end{enumerate}
		\end{example}
	
		\begin{definition}[Group isomorphism]
			\label{def:isomorphism}
			Let $\varphi: G \rightarrow H$ be a group homomorphism.
			If $\varphi$ is bijective, it is called a \emph{group isomorphism}.
			If there exists a group isomorphism between $G$ and $H$, then $G$ and $H$ are said to be \emph{isomorphic} and we write $G \cong H$.
		\end{definition}
	
		\begin{example}\leavevmode
			\begin{enumerate}[label=(\alph*)]
				\item The map $\varphi$ in \cref{exp:homomorphism}\labelcref{exp:homomorphism.exp} is bijective (why?) and is therefore an isomorphism.
				$(\symbb{R}, +, 0)$ and $(\symbb{R}^+, \times, 1)$ are therefore isomorphic.
				\item Let $\symcal{C}_4 = \left(\braket{\hat{C}_4 \mid \hat{C}^4_4 = \hat{E}}, \cdot, \hat{E}\right)$ be the cyclic group of order 4 and $G = \left(\{1, i, -1, -i\}, \cdot, 1\right)$ the group generated by the imaginary unit $i$.
				The group homomorphism $\varphi: G \rightarrow \symcal{C}_4$ given by
					\begin{equation*}
						\varphi: 1 \mapsto \hat{E}, \quad i \mapsto \hat{C}_4, \quad -1 \mapsto \hat{C}^2_4, \quad -i \mapsto \hat{C}^3_4
					\end{equation*}
				is a group isomorphism.
				Therefore, $G$ and $\symcal{C}_4$ are isomorphic.
				\item The point groups $\symcal{T}_d$ and $\symcal{O}$ are isomorphic.
			\end{enumerate}
		\end{example}
	
	
	\subsection{Vector Space}
	
		For many groups, we often think about their elements as linear transformations acting on some particular set of functions.
		To have a more rigorous understanding of this, we must digress momentarily from our discussion of groups and introduce the concept of a \emph{vector space}.
		
		\begin{definition}[Vector space]
			\label{def:vectorspace}
			Let $V$ be an (additive) abelian group and $\symbb{F}$ a commutative field.
			$V$ is called a \emph{vector space over $\symbb{F}$} if, for all elements $\symbfit{v}_1, \symbfit{v}' \in V$ and scalars $\lambda, \lambda' \in \symbb{F}$,
			\begin{enumerate}[label=(\roman*)]
				\item (distributivity) $\lambda(\symbfit{v}+\symbfit{v}') = \lambda\symbfit{v}+\lambda\symbfit{v}'$;
				\item (distributivity) $(\lambda+\lambda')\symbfit{v} = \lambda\symbfit{v}+\lambda'\symbfit{v}$;
				\item (associativity) $\lambda'(\lambda\symbfit{v}) = (\lambda'\lambda)\symbfit{v}$;
				\item (unity) $1\symbfit{v} = \symbfit{v}$.
			\end{enumerate}
		\end{definition}
		
		It is clear from the above \namecref{def:vectorspace} that neither a basis nor a concept of distances and angles is needed for a vector space.
		Nevertheless, it is often helpful to specify a vector space using a \emph{basis}.
		
		\begin{definition}[Basis]
			Let $V$ be a vector space over a field $\symbb{F}$.
			A subset $B$ of $V$ is called a \emph{basis} of $V$ if $B$ satisfies the following conditions:
			\begin{enumerate}[label=(\roman*)]
				\item (linear independence) for every finite subset $\{\symbf{e}_i \mid i=1,2,\ldots,n\}$ of $B$ and every $a_1, a_2, \ldots, a_n \in \symbb{F}$, if $\sum_{i=1}^{n}\symbf{e}_i a_i = 0$ then it must follow that $a_i = 0$ for all $i=1,2,\ldots,n$;
				\item (spanning) for every $\symbfit{v} \in V$, it is possible to choose $\symbf{e}_i \in B$ and $v_i \in \symbb{F}$ such that $\symbfit{v} = \sum_{i}\symbf{e}_i v_i$.
			\end{enumerate}
		\end{definition}
		
		\begin{remark}
			We commonly say that $B$ spans $V$ or $V$ is spanned by $B$.
			$B$ therefore provides a compact way of specifying $V$, and the number of elements of $B$ (the cardinality of $B$) is the dimension of $V$.
		\end{remark}
	
		In many vector spaces we encounter, it is often necessary to define an additional structure called \emph{inner product} so that the concept of distances and angles can be formalised, but it is important to recognise that not all vector spaces can be endowed with an inner product.
		
		\begin{definition}[Inner product]
			Let $V$ be a vector space over a field $\symbb{F}$.
			An \emph{inner product} on $V$ is a map $\braket{\cdot, \cdot}: V \times V \rightarrow \symbb{F}$ that associates each pair of vectors to a scalar quantity and satisfies the following properties:
			\begin{enumerate}[label=(\roman*)]
				\item (conjugate symmetry) for all $v_i, v_j \in V$, $\braket{\symbfit{v}_i, \symbfit{v}_j} = \braket{\symbfit{v}_j, \symbfit{v}_i}^*$;
				\item (linearity in the first argument) for all $v_i, v_j \in V$ and $a \in \symbb{F}$, $\braket{a\symbfit{v}_i, \symbfit{v}_j} = a\braket{\symbfit{v}_i, \symbfit{v}_j}$;
				\item (positive-definiteness) for all $v_i \in V$, $\braket{\symbfit{v}_i, \symbfit{v}_i} \ge 0$ and $\braket{\symbfit{v}_i, \symbfit{v}_i} \ge 0 \Leftrightarrow \symbfit{v}_i = \symbf{0}$.
			\end{enumerate}
		\end{definition}
		
		We often think about a vector space $V$ as the space of geometrical vectors in some dimensions, due mainly to its name, but $V$ can be much more general than that.
		The following definitions and examples of some concrete vector spaces should make this clear.
		
		\begin{definition}[Coordinate space]
			For a positive integer $n$, the set of all $n$-tuples of elements of $\symbb{F}$, $\symbfit{v} = (v_1, v_2, \ldots, v_n)$,
			forms an $n$-dimensional vector space over $\symbb{F}$ called a \emph{coordinate space} denoted by $\symbb{F}^n$.
		\end{definition}
	
		\begin{remark}\leavevmode
			\begin{enumerate}[label=(\roman*)]
				\item A standard basis for $\symbb{F}^n$ is
					\begin{align*}
						\symbf{e}_1 &= (1, 0, \ldots, 0)\\
						\symbf{e}_2 &= (0, 1, \ldots, 0)\\
						\dots\\
						\symbf{e}_n &= (0, 0, \ldots, 1).
					\end{align*}
					
				\item Consider $\symbb{F} = \symbb{R}$.
				The coordinate space $\symbb{R}^n$ by itself has no inner products associated to it.
				However, if we define an inner product on this space such that, for any $\symbfit{u} = (u_1, u_2, \ldots, u_n)$ and $\symbfit{v} = (v_1, v_2, \ldots, v_n)$,
					\begin{equation*}
						\braket{\symbfit{u}, \symbfit{v}} = \sum_{i=1}^{n} u_i v_i,
					\end{equation*}
				then $\symbb{R}^n$ becomes a \emph{Euclidean space}.
				The inner product $\braket{\symbfit{u}, \symbfit{v}}$ as defined above is called the \emph{dot product} of $\symbfit{u}$ and $\symbfit{v}$.
			\end{enumerate}
		\end{remark}
		
		\begin{definition}[Function space]
			Let $X$ be a set.
			The set of all maps $f: X \rightarrow \symbb{F}$ forms a vector space over $\symbb{F}$ called a \emph{function space} and denoted $\symbb{F}^X$.
		\end{definition}
		
		\begin{example}\leavevmode
			\begin{enumerate}[label=(\alph*)]
				\item The set of all maps $f: \symbb{R} \rightarrow \symbb{R}$ given by $f: x \mapsto \sum_{i=0}^{n}x^i a_i$ for $a_i \in \symbb{R}$ is a function space of polynomials of degrees no more than $n$.
				This space is spanned by the set of monomials $\{x^i \mid i=0,1,\ldots,n\}$.
				One possible inner product in this space is $\braket{f, g} = \int_{0}^{1}f(x)g(x) \D x$.
				\item The set of all maps $f: \symbb{R} \times \symbb{R} \rightarrow \symbb{C}$ given by $f: (\theta, \phi) \mapsto \sum_{l=0}^{n} \sum_{m=-l}^{l} Y_{l}^{m}(\theta, \phi)\ a_{l}^{m}$ for $a_{l}^{m} \in \symbb{R}$ and $Y_{l}^{m}$ a spherical harmonic function of degree $l$ and order $m$ is a function space.
				This space is spanned by the set of spherical harmonics $\{Y_{l}^{m} \mid l=0,1,\ldots,n; \ m=-l,-l+1,\ldots,l\}$.
				One possible inner product in this space is $\braket{f, g} = \int_{\phi=0}^{2\pi} \int_{\theta=0}^{\pi} f^*(\theta, \phi)g(\theta, \phi) \sin\theta \D\theta\D\phi$.
			\end{enumerate}
		\end{example}

		In anticipation of what will be explained in {\color{red} on representation theory}, whenever we think about the representation of a certain symmetry operation as a matrix, this matrix always has to be written with respect to a certain basis spanning a particular vector space.
		A basis is therefore crucial in the specification of any representation.


	\subsection{Some Important Abstract Groups}
	
		We can now get back to our discussion of group theory where we will provide a quick survey of some important abstract groups to which many of the commonly encountered concrete groups are isomorphic, and quite a few of which have far-reaching implications in quantum mechanics.
		The abstractness of these groups admittedly does make them quite hard to be understood at first, but having at least a notion of them helps us think about the zoo of finite and infinite groups more systematically.
		
		\subsubsection{Linear Groups}
			\label{sec:lineargroups}
			
			A lot of symmetry operations are actually linear maps acting on a certain vector space.
			The set of all such maps turns out to form a very large group which we will soon defined to be the \emph{general linear group}.
			We first introduce this group in its most abstract sense, and then relate it to more concrete groups of matrices via group isomorphisms.
			Furthermore, this group plays a crucial role in the formalism of representation theory, and having at least a brief acknowledgement of its existence does help clarify quite a lot of ideas in the representations of groups which we will explore later on.
			
			\begin{proposition}
				Let $V$ be a vector space over a field $\symbb{F}$.
				The set of all invertible linear maps $\theta: V \rightarrow V$ forms a group.
			\end{proposition}
		
			\begin{proof}
				This proof is left as an exercise to the reader.
			\end{proof}
		
			\begin{definition}[General linear group]
				\label{def:GL}
				Let $V$ be a vector space over a field $\symbb{F}$.
				The group of all invertible linear maps $\theta: V \rightarrow V$ is called the \emph{general linear group} of $V$ and denoted by $\symup{GL}(V)$.
			\end{definition}
			
			\begin{remark}\leavevmode
				\begin{enumerate}[label=(\alph*)]
					\item The binary operation in $\symup{GL}(V)$ is the operation of function composition, $\circ$, where $(\theta_2 \circ \theta_1)(\symbfit{v}) = \theta_2\left[\theta_1(\symbfit{v})\right]$ for some $\symbfit{v} \in V$.
					\item The identity of $\symup{GL}(V)$ is the identity map $\symup{id}: V \rightarrow V$ defined as $\symup{id}: \symbfit{v} \mapsto \symbfit{v}$ which maps each $\symbfit{v} \in V$ to itself.
				\end{enumerate}
			\end{remark}
		
			\Cref{def:GL} does seem to make $\symup{GL}(V)$ look like a very abstract group.
			It is indeed a very abstract group, and it might therefore be easier to think about $V$ as the space of geometric vectors that we are familiar with, and $\symup{GL}(V)$ as the group of all invertible linear transformations mapping each vector to another vector in $V$.
			
			Now, if $V$ is a vector space of $n$ dimensions over $\symbb{F}$, then we can choose a basis $\left\lbrace \symbf{e}_1, \symbf{e}_2, \ldots, \symbf{e}_n \right\rbrace$ for $V$ over $\symbb{F}$.
			Then, every invertible linear transformation $\theta \in \symup{GL}(V)$ corresponds to an $n \times n$ invertible matrix $\symbfit{M}_{\theta} = [M_{ij}(\theta)]$ given by
				\begin{equation*}
					\theta(\symbf{e}_j) = \sum_{i=1}^{n} \symbf{e}_i M_{ij}(\theta).
				\end{equation*}
			This motivates the following.
			
			\begin{proposition}
				The set of $n \times n$ invertible matrices over $\symbb{F}$, together with the binary operation of ordinary matrix multiplication, forms a group.
				We denote this group $\symup{GL}_n(\symbb{F})$.
			\end{proposition}
		
			\begin{proof}
				This proof is left as a trivial exercise to the reader.
			\end{proof}
		
			\begin{proposition}
				$\symup{GL}(V)$ is isomorphic to $\symup{GL}_n(\symbb{F})$ with the group isomorphism given by $\theta \mapsto \symbfit{M}_{\theta}$ where $\symbfit{M}_{\theta}$ are $n \times n$ invertible matrices.
			\end{proposition}
		
			\begin{proof}
				This proof is left as another (slightly more involved) exercise to the reader.
				In essence, we need to show that $\theta: \symup{GL}(V) \rightarrow \symup{GL}_n(\symbb{F})$ is a group homomorphism, and then that $\theta$ is bijective.
			\end{proof}
		
			It is important to remember that $\symup{GL}(V)$ is defined as an abstract group of linear maps acting on some vector space, whereas $\symup{GL}_n(\symbb{F})$ is a concrete group of matrices.
			However, the isomorphism between the two enables us to think about one via the other.
			We will have a lot more to say about these groups when we introduce representation theory.

			

%		\subsubsection{Linear Groups}
%			\label{sec:lineargroups}